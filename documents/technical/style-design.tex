\section{Требования к стилю и цветовой схеме}

По возможности, все элементы \emph{UI} должны являться стандартными android компонентами для уменьшения времени разработки и уменьшения вероятности создания ненужных ошибок. Все элементы, независимо от способа их получения, должны соответствовать \emph{Google material design}~\cite{design}.

Все кнопки, поля текстового ввода, списки и их элементы являются плашками. Это значит что у них нет границ, они имеют только тень. Минимальное расстояние между любыми двумя плашками --- это 2 пикселя, независимо от размеров плашек и размера экрана устройства. Плашки на одном уровне не кидают на друг друга тень. Например все элементы списка находятся на одном уровне, но сам список находится ниже, из-за этого элементы списка бросают тень на подложку, но не на друг друга.

Приложение должно иметь стандартную цветовую схему от Google~\cite{colors}. Цвета данной темы представлены в таблице~\ref{tbl:theme}.

\begin{table}[H]
    \centering
    \caption{\label{tbl:theme}Color theme}
    \begin{tabular}{|c|c|c|c|}
        \hline
        Type & Color [hex] & On Type & On Color [hex]\\
        \hline
        Primary & 0x6200EE & On Primary & 0xFFFFFF\\
        Secondary & 0x03DAC5 & On Secondary & 0x000000\\
        Background & 0xFFFFFF & On Background & 0x000000\\
        Surface & 0xFFFFFF & On Surface & 0x000000\\
        Error & 0xB00020 & On Error & 0x000000\\
        \hline
    \end{tabular}
\end{table}
