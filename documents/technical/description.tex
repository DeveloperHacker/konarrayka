\section{Подробное описание приложения}

В данном разделе будут подробно описаны все экраны приложения и их назначение.

\section{Начальный экран}\label{lbl:start}

\subsection{Эскиз \emph{3.a}}

\begin{enumerate*}
    \item кнопка для перехода к параметрам приложения (см. экран~\ref{lbl:app-settings});
    \item кнопка перехода к предыдущей игре в списке, не отображается, если данная играя является первой в списке;
    \item кнопка перехода к следующей игре;
    \item имя игры;
    \item изображение игры.
\end{enumerate*}
Нажатие на данный экран приводит к переходу на экран сетапа соответствующей игры (см. экран~\ref{lbl:game-setup}), в случае если в конфигурации игры есть ошибки, то игра не может быть запущена и должно выводится предупреждение с двумя кнопками \emph{fix} и \emph{cancel}. Нажатие на кнопку \emph{fix} должно переводить на экран с ошибкой для её исправления, также должен устанавливаться фокус на тот элемент, при помощи которого можно решить данную проблему. Например если пользователь не указал длительность действия игры, то должен произойти переход на данный экран, также данное поле должно быть видимо для пользователя и должно быть выведено pop-up сообщение с ошибкой. Долгое нажатие приводит к редактированию конфигурации данной игры (см. экран~\ref{lbl:game-new}). Свайп вправо --- переход к предыдущей игре в списке, свайп влево --- следующей.

\subsection{Эскиз \emph{3.b}}

\begin{enumerate*}
    \item кнопка для перехода к параметрам приложения (см. экран~\ref{lbl:app-settings});
    \item изображение символизирующее отсутствие игры;
    \item описание предназначения данного элемента в списке игр;
    \item кнопка перехода к предыдущей игре в списке, не отображается, если в списке нет игр.
\end{enumerate*}
Нажатие на данный экран приводит к переходу на экран выбора способа получения новой игры (см. экран~\ref{lbl:choose}) (создание новой (см. экран~\ref{lbl:game-new}) или поиск в файловой системе (см. экран~\ref{lbl:browse})). Свайп вправо --- переход к последней игре в списке.

\section{Экран параметров приложения}\label{lbl:app-settings}

\subsection{Эскиз \emph{3.с}}

\begin{enumerate*}
    \item применение настроек и возврат к предыдущему экрану (из которого были вызваны данные настройки);
    \item шкала общей громкости приложения (звук нажатия клавиш, озвучка, музыка и тп);
    \item шкала громкости музыки в игре;
    \item имя шкалы музыки;
    \item имя шкалы звука;
    \item имя маркера вибрации;
    \item маркер присутствия вибрации в игре;
\end{enumerate*}

\section{Сетап игры}\label{lbl:game-setup}

\subsection{Эскиз \emph{3.e}}

\begin{enumerate*}
    \item кнопка для перехода к параметрам игры (см. экран~\ref{lbl:game-settings});
    \item кнопка для перехода на экран выбора персонажей (см. экран~\ref{lbl:game-characters});
    \item кнопки для перехода на экран выбора персонажей (см. экран~\ref{lbl:game-characters}) с уже выбранными персонажами для игры;
    \item отображение числа игроков (не обязано совпадать с количеством персонажей в игре) для соответствующего сетапа игры.
\end{enumerate*}

\section{Экран параметров игры}\label{lbl:game-settings}

\subsection{Эскиз \emph{3.d}}

\begin{enumerate*}
    \item применение настроек и возврат к предыдущему экрану (из которого были вызваны данные настройки);
    \item имя поля с фоновой музыкой в игре;
    \item кнопка для перехода к выбору мелодии (см. экран~\ref{lbl:sound-view}), которая будет играть во время игры (RANDOM --- случайная мелодия из списка заданного разработчиками конфигураций игры). Текст внутри кнопки отображает имя выбранной мелодии;
    \item имя переменной времени, разработчики конфигурации игры должны позаботится о том, чтобы это имя отражало назначение переменной времени;
    \item поле со значением переменной времени;
    \item маркер единиц измерения времени;
    \item обозначение того, что данная страница должна скролиться в случае если все переменные не влезают в один экран.
\end{enumerate*}

\section{Экран выбора звука}\label{lbl:sound-view}

\subsection{Эскиз \emph{3.d}}

\begin{enumerate*}
    \item возврат к предыдущему экрану (из которого был запрошен данный список);
    \item имя списка со звуками;
    \item обозначение того, что в игре будет воспроизводится случайная мелодия из данного списка;
    \item имя мелодии.
\end{enumerate*}

\section{Экран выбора персонажей игры}\label{lbl:game-characters}

\subsection{Эскиз \emph{3.f}}

\begin{enumerate*}
    \item кнопка для перехода к параметрам игры (см. экран~\ref{lbl:game-settings});
    \item имя списка с персонажами для игры;
    \item иконка персонажа игры;
    \item выделенная иконка персонажа игры (данный персонаж будет присутствовать в игре). Данный ободок имеет размер в 2 пикселя квадрат на 1 пиксель выступает по обе стороны края иконки персонажа;
    \item кнопка для запуска игры (переход к первому действию игры (см. экран~\ref{lbl:game-action}));
\end{enumerate*}

\section{Экран действия игры}\label{lbl:game-action}

Вид экрана действия игры зависит от присутствия изображения, описания и его времени. Если задан только один элемент из данного перечня, то экран соответствует эскизу \emph{5.b}, иначе \emph{5.a}. Время является не заданным, если оно задано относительно длительности записи диктора (см. экран задания переменной времени~\ref{lbl:game-time-var})). При возврате на экран выбора персонажей (см. экран~\ref{lbl:game-characters}), на нём должны быть предварительно отмечены герои, которые участвовали в данной игре.

\subsection{Эскиз \emph{5.a}}

\begin{enumerate*}
    \item кнопка для перехода к параметрам приложения (см. экран~\ref{lbl:app-settings});
    \item место для отображения иконки действия, в случае если она была задана, иначе отображение его описания;
    \item место для отображения текста, в случае если если он был задан и задана иконка, иначе времени до конца данного действия;
    \item кнопка для досрочного завершения игры и переход на экран выбора персонажей (см. экран~\ref{lbl:game-characters});
    \item кнопка досрочного завершения действия для перехода к следующему или переход на экран выбора персонажей (см. экран~\ref{lbl:game-characters}), если это было последнее действие.
\end{enumerate*}

\subsection{Эскиз \emph{5.b}}

\begin{enumerate*}
    \item кнопка для перехода к параметрам приложения (см. экран~\ref{lbl:app-settings});
    \item место для отображения иконки, описания или времени до конца действия в зависимости от их присутствия;
    \item кнопка для досрочного завершения игры и переход на экран выбора персонажей (см. экран~\ref{lbl:game-characters});
    \item кнопка досрочного завершения действия для перехода к следующему или переход на экран выбора персонажей (см. экран~\ref{lbl:game-characters}), если это было последнее действие.
\end{enumerate*}

\section{Экран выбора: создание или поиск}\label{lbl:choose}

Фоном данного экрана является затенённый предыдущий экран. На данный экран невозможно вернуться, если это происходит, то возврат производится на экран, который предшествовал данному (который является задним планом).

\subsection{Эскиз \emph{5.d}}

\begin{enumerate*}
    \item кнопка для перехода на экран создания соответствующего компонента (конфигурации игры (см. экран~\ref{lbl:game-new}) или записи звука (см. экран~\ref{lbl:sound-record}));
    \item кнопка для поиска соответствующего файла в памяти телефона (см. экран~\ref{lbl:browse}).
\end{enumerate*}

\section{Экран поиска фалов}\label{lbl:browse}

\subsection{Эскиз \emph{4.e}}

\begin{enumerate*}
    \item кнопка для возврата на предыдущий экран;
    \item имя списка предложений;
    \item кнопка для перехода к ручному поиску фала (см. экран~\ref{lbl:browse-custom});
    \item кнопки для выбора файла из списка предложений и последующего возврата на предыдущий экран;
\end{enumerate*}

\section{Экран ручного поиска}\label{lbl:browse-custom}

\subsection{Эскиз \emph{4.f}}

\begin{enumerate*}
    \item кнопка для подтверждения результата и возврата на предыдущий экран;
    \item имя поля для ручного задания пути к файлу;
    \item поле для ручного задания пути к файлу.
\end{enumerate*}

\section{Экран записи звука}\label{lbl:sound-record}

\subsection{Эскиз \emph{5.c}}

\begin{enumerate*}
    \item кнопка для подтверждения результата записанного звука и возврата на предыдущий экран;
    \item поле для отображения длительности записи;
    \item кнопка для запуска записи, становится не кликабельной во время записи или воспроизведения;
    \item кнопка для остановки записи, изначально не кликабельна, изменяется на кнопку воспроизведения \faPlay, во время время воспроизведения сменяется на кнопку остановки записи \faStop; 
\end{enumerate*}

\section{Экран создания новой игры}\label{lbl:game-new}

Возврат на главный экран должен приводить к полной проверке конфигурации игры, в случае нахождения каких либо ошибок или недостатков, должно выводится предупреждающее сообщение с двумя кнопками \emph{save} и \emph{fix}. Нажатие кнопки \emph{fix} должно работать также, как в на начальном экране (см. экран~\ref{lbl:start}) В случае если пользователь нажал кнопку \emph{save}, то конфигурации должны сохранится независимо от наличия ошибок.

\subsection{Эскиз \emph{1.f}}
\begin{enumerate*}
    \item кнопка для сохранения конфигураций игры и переход в главное меню;
    \item обозначение того, что данный экран является скролящимся;
    \item декоративное наименование;
    \item поле с изображением игры, нажатие на него должно приводить к переходу на экран поиска в файловой системе (см. экран~\ref{lbl:browse})
    \item имя поля для ввода имени игры;
    \item поле для ввода имени игры. Значение поля по умолчанию ``untitled'', которое не отображается, в случае если не было задано специально;
    \item имя кнопки для перехода к заданию правил игры;
    \item кнопка для перехода к заданию правил игры. После сохранения правил игры, внутри данной кнопки должна появится надпись соответствующая шаблону: ``<game-name>.scn'';
    \item наименование списка с фоновой музыкой игры;
    \item плашка с именем музыки, которая будет использоваться для фонового сопровождения во время игры;
    \item скролящийся список с фоновой музыкой для игры;
    \item кнопка для добавления звука в данной список, посредством экрана выбора (см. экран~\ref{lbl:choose}). Новая музыка создаётся при помощи экрана записи~(см. экран~\ref{lbl:sound-record});
    \item наименование кнопки добавления конфигураций сетапа игры.
    \item кнопка для перехода к конфигурированию сетапов игры (см. экран~\ref{lbl:game-setups-new}). После сохранения сетапа игры, внутри данной кнопки должна появится надпись соответствующая шаблону: ``<game-name>.stp'';
\end{enumerate*}

\section{Задание правил игры}\label{lbl:game-scenario-new}

Быстрое нажатие на любой из свёрнутых столбцов приводит к разворачиванию данного и сворачиванию остальных (см. экраны \emph{1.a}, \emph{1.b}, \emph{1.c}, \emph{1.e}). Долгое нажатие на любую из плашек приводит к переходу на экран редактирования данной плашки в зависимости от её типа. Удержание и движение плашки с действием или персонажем приводит к её переносу независимо от того раскрыта колонка или нет, во время переноса плашки автоматически разворачивается самая правая колонка, задающая порядок и условия действий (см. экран \emph{1.e}). Наведение и отпускание плашки персонажа на действие из правой колонки приводит к добавлению персонажа в условие данного действия (см. экран задания условия действия~\ref{lbl:game-action-cond}). В случае если при добавлении персонажа в условие действия, данный персонаж присутствует в последнем конъюнкте, то он не добавляется в список и выводится pop-up сообщение: ``character already added''.

\subsection{Эскиз \emph{1.a}}

Отображение развёрнутый столбец последовательности действий игры.

\begin{enumerate*}
    \item кнопка для добавления нового действия, переход на экран задания нового действия (см. экран ~\ref{lbl:game-action-new});
    \item кнопка для добавления нового персонажа, переход на экран задания нового персонажа (см. экран ~\ref{lbl:game-character-new});
    \item кнопка отмены последнего изменения добавление нового действия, изменение порядка действий и тп;
    \item кнопка сохранения конфигурации правил игры и переход к экрану создания игры (см. экран ~\ref{lbl:game-new});
    \item плашка действия с привязанным именем;
    \item плашка персонажа с привязанным именем и маркером цвета;
    \item маркер цвета персонажа;
    \item маркер цвета персонажа привязанный к действию, обозначающий присутствие данного персонажа в условии появления данного действия;
    \item кнопка для добавления действия в конец списка, переход на экран выбора действия (см. экран ~\ref{lbl:game-action-list}).
\end{enumerate*}

\subsection{Эскиз \emph{1.b}}

Отображение развёрнутого столбца созданных действий.

\begin{enumerate*}
    \item развёрнутый столбец с плашками действий;
    \item свёрнутый столбец с плашками персонажей;
    \item свёрнутый столбец последовательности действий игры;
    \item кнопка для добавления нового действия, переход на экран задания нового действия (см. экран ~\ref{lbl:game-action-new});
    \item кнопка для добавления нового персонажа, переход на экран задания нового персонажа (см. экран ~\ref{lbl:game-character-new});
    \item кнопка сохранения конфигурации правил игры и переход к экрану создания игры (см. экран ~\ref{lbl:game-new});
    \item кнопка отмены последнего изменения добавление нового действия, изменение порядка действий и тп;
    \item кнопка для добавления действия в конец списка, переход на экран выбора действия (см. экран ~\ref{lbl:game-action-list}).
\end{enumerate*}

\subsection{Эскиз \emph{1.c}}

Отображение развёрнутого столбца созданных персонажей.

\begin{enumerate*}
    \item свёрнутый столбец с плашками действий;
    \item развёрнутый столбец с плашками персонажей;
    \item свёрнутый столбец последовательности вызова действий игры;
    \item кнопка сохранения конфигурации правил игры и переход к экрану создания игры (см. экран ~\ref{lbl:game-new});
    \item кнопка отмены последнего изменения добавление нового действия, изменение порядка действий и тп;
    \item кнопка для добавления нового действия, переход на экран задания нового действия (см. экран ~\ref{lbl:game-action-new});
    \item кнопка для добавления нового персонажа, переход на экран задания нового персонажа (см. экран ~\ref{lbl:game-character-new});
    \item кнопка для добавления действия в конец списка, переход на экран выбора действия (см. экран ~\ref{lbl:game-action-list}).
\end{enumerate*}

\subsection{Эскиз \emph{1.e}}

Отображение вставки действия между другими действиями столбце упорядоченных действий.

\begin{enumerate*}
    \item визуальное предоставление места для вставки действия;
    \item действие перенесённое пользователем для вставки;
    \item кнопка сохранения конфигурации правил игры и переход к экрану создания игры (см. экран ~\ref{lbl:game-new});
    \item кнопка отмены последнего изменения добавление нового действия, изменение порядка действий и тп;
    \item кнопка для добавления нового действия, переход на экран задания нового действия (см. экран ~\ref{lbl:game-action-new});
    \item кнопка для добавления нового персонажа, переход на экран задания нового персонажа (см. экран ~\ref{lbl:game-character-new});
    \item кнопка для добавления действия в конец списка, переход на экран выбора действия (см. экран ~\ref{lbl:game-action-list}).
\end{enumerate*}

\section{Экран конфигуриования сетапов игры}\label{lbl:game-setups-new}

\subsection{Эскиз \emph{2.c}}
\begin{enumerate*}
    \item число игроков, на которое рассчитан данный сетап;
    \item кнопка сохранения сетапов игры и возврат к экрану конфигурирования игры (см. экран~\ref{lbl:game-new});
    \item иконки персонажей, которые присутствуют в данном сетапе;
    \item кнопка перехода к конфигурированию нового сетапа (см. экран~\ref{lbl:game-setup-new});
    \item наименование списка с разными сетапами игры.
\end{enumerate*}

\section{Экран создания сетапа игры}\label{lbl:game-setup-new}

\subsection{Эскиз \emph{2.e}}
\begin{enumerate*}
    \item кнопка сохранения сетпа игры;
    \item наименование поля для задания расчётного количества игроков для данного сетапа;
    \item поле для задания расчётного количества игроков для данного сетапа, по умолчанию равно количеству персонажей;
    \item наименование списка с персонажами из игры;
    \item маркер означающий присутствие данного персонажа в сетапе;
    \item персонаж без отметки.
\end{enumerate*}

\section{Экран выбора действия игры}\label{lbl:game-action-list}

\subsection{Эскиз \emph{4.a}}
\begin{enumerate*}
    \item кнопка для возврата на предыдущий экран;
    \item наименования списка с действиями;
    \item плашка действия, при нажатии на неё производится возврат на предыдущий экран с выбранным действием.
\end{enumerate*}

\section{Экран задания условия действия игры}\label{lbl:game-action-cond}

\subsection{Эскиз \emph{1.d}}
\begin{enumerate*}
    \item кнопка для возврата на предыдущий экран;
    \item плашка с логическим оператором \emph{ИЛИ}, по умолчанию между всеми персонажами стоит логическое \emph{И}. Данная запись является дизъюнктивная нормальная форма, относительно присутствия или отсутствия персонажей в игре;
    \item плашка с персонажем;
    \item кнопка для добавления персонажа в конец списка. Для этого производится переход на экран~\ref{lbl:game-character-list}. В случае если при добавлении персонажа в условие действия, данный персонаж присутствует в последнем конъюнкте, то он не добавляется в список и выводится pop-up сообщение: ``character already added''.
\end{enumerate*}

\section{Экран выбора персонажа игры}\label{lbl:game-character-list}

\subsection{Эскиз \emph{4.b}}
\begin{enumerate*}
    \item кнопка для возврата на предыдущий экран;
    \item наименования списка с персонажами;
    \item плашка с логическим оператором \emph{ИЛИ};
    \item плашка с персонажем;
    \item цвет соответствующего персонажа.
\end{enumerate*}

Нажатие на плашку приводит к переходу на предыдущий экран с выбранным персонажем.

\section{Экран задания действия игры}\label{lbl:game-action-new}

\subsection{Эскиз \emph{5.e}}
\begin{enumerate*}
    \item сохранение конфигураций действия и переход на предыдущий экран;
    \item декоративное наименование действия;
    \item поле с изображением действия игры, нажатие на него должно приводить к переходу на экран поиска в файловой системе (см. экран~\ref{lbl:browse});
    \item наименование поля для задания имени действия;
    \item поле для задания имени действия, по умолчанию ``action<uniq-numebr>'';
    \item наименование поля описания дйствия;
    \item поле для описания действия;
    \item наименование списка озвучек данного действия;
    \item плашка с именем файла озвучки;
    \item список озвучек данного действия;
    \item кнопка добавления звука в список озвучек, посредством экрана выбора (см. экран~\ref{lbl:choose}). Новая музыка создаётся при помощи экрана записи~(см. экран~\ref{lbl:sound-record});
    \item наименование поля длительности действия;
    \item поле для задания длительности действия, долгое нажатие приводит к переходу на экран выбора переменных времени (см. экран~\ref{lbl:game-time-var}) для привязки данной длительности к переменной;
    \item наименование единицы измерения длительности действия.
\end{enumerate*}

\section{Экран задания персонажа игры}\label{lbl:game-character-new}

\subsection{Эскиз \emph{2.a}}
\begin{enumerate*}
    \item сохранение конфигураций персонажа и переход на предыдущий экран;
    \item декоративное наименование персонажа;
    \item поле с изображением персонажа, нажатие на него должно приводить к переходу на экран поиска изображения в файловой системе (см. экран~\ref{lbl:browse});
    \item наименование поля для ввода имени персонажа;
    \item поле для задания имени персонажа, значение по умолчанию ``character<uniq-numebr>'';
    \item наименования поля для задания цвета персонажа;
    \item кнопка для перехода к экрану выбора цвета (см. экран~\ref{lbl:game-color}). по умолчанию равно мало использованному цвету. Цвет кнопки зависит от цвета персонажа. Долгое нажатие приводит к заданию цвета используя \emph{hex} код.
\end{enumerate*}


\section{Экран выбора цвета}\label{lbl:game-color}

\subsection{Эскиз \emph{2.b}}
\begin{enumerate*}
    \item кнопка возврата на предыдущий экран;
    \item декоративное наименование списка цветов;
    \item плашка с программно заданным цветом.
\end{enumerate*}

\section{Экран задания переменной времени}\label{lbl:game-time-var}

\subsection{Эскиз \emph{2.f}}

\begin{enumerate*}
    \item кнопка для возврата на предыдущий экран и сохранения созданных переменных;
    \item значение по умолчанию для соответствующей переменной;
    \item имя переменной времени, данное имя будет отображаться в настройках игры;
    \item кнопка для добавления новых переменных времени c переходом на экран создания~\ref{lbl:game-time-var-new};
    \item имя списка переменных;
    \item переменная для задания времени действия равная длительности звука диктора;
\end{enumerate*}
Удаление переменных времени производится свайпом вправо или влево. Переменную \emph{SOUND} невозможно удалить. Долгое нажатие на переменную времени переводит на экран редактирования переменной.  Любое изменение имени переменной приводит к изменению имени в внутри действий. Удаление переменной приводит к очищению поля длительности действия. Следует вывести предупреждающее сообщение с двумя кнопками \emph{accept} и \emph{decline}.

\section{Экран создания переменной времени}\label{lbl:game-time-var-new}

\subsection{Эскиз \emph{3.d}}

\begin{enumerate*}
    \item кнопка для подтверждения результата и возврата на предыдущий экран;
    \item имя поля для задания имени переменной;
    \item поле для задания имени переменной;
    \item имя поля для задания значения по умолчанию;
    \item поле для задания значения по умолчанию;
\end{enumerate*}


