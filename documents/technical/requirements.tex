\section{Технические требования}

Разработанное Android приложение должно обеспечивать возможность выполнения таких функций, как:

\begin{itemize*}
    \item Запуск на мобильных устройствах с версией Android API не ниже 28 (9.0 Pie);
    \item Задание правил игры, используя встроенные возможности приложения;
    \item Сохранение заданных правил игры в память мобильного устройства;
    \item Генерация сценария на основе заданных правил игры;
    \item Конфигурирование одного выбранного сценария игры;
    \item Сохранение конфигураций одиночного сценария игры в память мобильного устройства;
    \item Воспроизведение сконфигурированного сценария игры.
\end{itemize*}

\subsection{Конфигурирование правил игры}

Конфигурирование правил игры должно происходить в несколько этапов с возможностью перехода к предыдущему: 

\begin{enumerate*}
    \item определение фонового музыкального сопровождения;
    \item конфигурирование персонажей игры. При описании персонажа должна быть возможность задания его имени, иконки, текстового и звукового описания;
    \item конфигурирование действия. При описании действия должна быть возможность задания имени, текстового и звукового описания, продолжительности действия, персонажа или персонажей, к которым привязано данное действие. Также должна быть возможность задания времени действия во время конфигурирования одиночного сценария игры;
    \item определение порядка выполнения действий, с возможностью задания условия появления действия в конечном сценарии игры. Условие появления действия должно зависеть от присутствия или отсутствия заданных персонажей в игре;
    \item определение зависимостей присутствия между персонажами в игре;
    \item определение сценариев по умолчанию в зависимости от количества игроков, которое будет задано во время конфигурирования одиночного сценария игры.
\end{enumerate*}

\subsection{Конфигурирование сценария игры}

Конфигруирование сценария игры должно происходить перед запуском игры. В случае если существуют настройки по умолчанию, приложение должно спросить количество игроков в игре. К конфигурируемым параметрам относятся: длительности действий в игре и набор персонажей участвующих в игре. После задания параметров используемого сценария игры должна пройти проверка на правильность выбора персонажей.

\subsection{Права доступа}

Многие возможности будут требовать специальные права доступа к компонентам мобильного устройства. Получение данных прав должно происходить по необходимости и быть прозрачным для пользователя. То есть пользователь должен понимать для чего приложение запрашивает права доступа к тому или иному компоненту.

\subsection{Сохранение в память мобильного устройства}

Сохранение конфигураций в память мобильного устройства должно сопровождаться такими возможностями как автоматическое и ручное задание пути для сохраняемого/читаемого конфигурационного файла.

\subsection{Работа со звуком}

Пользователь должен иметь два способа задания звуковой дорожки: через чтение звукового файла из памяти устройства или получение при помощи записи через микрофон используемого мобильного устройства.
